% Chapter 6

\chapter{Conclusion} % Main chapter title

\label{Chapter6}

\lhead{Chapter 6. \emph{Conclusion}} % This is for the header on each page - perhaps a shortened title

%----------------------------------------------------------------------------------------

% \emph{I konklusjonen din, beskriv status for ditt arbeid. Oppsummer hva du har oppnådd, sammenlignet med hva du opprinnelig ønsket å oppnå. Relater arbeidet til tidligere relevant arbeid. Foreslår videre arbeid som du tror vil være verdt.}

As was shown from the two test runs in chapter~\ref{Chapter5}, prediction results are very unreliable. The numbers describe a recommendation system with little or no overall connection to real-world ratings.

The system is able to achieve a low MAE for top movies because its predictions happen to average around the average benchmark rating. It is completely unable to consistently distinguish the quality of famous top-rated movies from unknown B-films. Furthermore, it is unable to process around half of the movie titles due to a lack of Twitter search results.

\section{Suggestions for further work} % (fold)
\label{sec:suggestions_for_further_work}

Although the sentiment analysis approach taken in this project fails, my impression of Twitter as being an extremely interesting data source has not been weakened. Its qualities, as described in section~\ref{sec:twitter_data}, are shared with very few other data sources (if any), and its API is both efficient, well-designed and powerful.

There are several improvements to the techniques employed in this project that are possible to explore. Noisy data having been identified as the main weakness of the processed data, it would be interesting to see an approach where the Twitter messages are filtered before sentiment analysis is initiated. A conditional random field (CRF) approach would be especially interesting to explore, being an active area of research in the field of opinion mining~\cite{Choi:2005:ISO:1220575.1220620, Jakob:2010:EOT:1870658.1870759, Yang:2012:EOE:2390948.2391100}. Other improvements may be applied to the sentiment classification step itself, where several other approaches report good results~\cite{agarwal2011sentiment, go2009twitterdistant, go2009twitter, kouloumpis2011twitter, pak2010twitter}.

One could also look into using the data in other ways than predicting real-numbered ratings. Extracting more descriptive characteristics about a movie could serve as an interesting approach, for instance by extracting sentiment-carrying adjectives like ``hilarious'', ``gritty'', ``exciting'' etc. from the search results.

Also worth noting are the ways in which the specific techniques used in this project can be applied to other data sources. Building on the original motivation in section~\ref{sec:motivation} and the current fact that the Twitter search API only returns data from the last 6-9 days (see section~\ref{sub:novelty_of_available_data}), the techniques might work better predicting ratings for new movies.

% section suggestions_for_further_work (end)
