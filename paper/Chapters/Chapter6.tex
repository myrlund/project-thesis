% Chapter 6

\chapter{Conclusion} % Main chapter title

\label{Chapter6}

\lhead{Chapter 6. \emph{Conclusion}} % This is for the header on each page - perhaps a shortened title

%----------------------------------------------------------------------------------------

% \emph{I konklusjonen din, beskriv status for ditt arbeid. Oppsummer hva du har oppnådd, sammenlignet med hva du opprinnelig ønsket å oppnå. Relater arbeidet til tidligere relevant arbeid. Foreslår videre arbeid som du tror vil være verdt.}

I thought that the data would contain too much noise to be usable in any other way than to annotate and -- in the best of cases -- adjust ratings of content where the social sentiment disagreed strongly with the proposed rating.

However, for certain kinds of content, the predictions generated by Twitter were more or less spot on the same as the ones from Netflix, as shown in chapter~\ref{Chapter5}.

This leads me to believe that Twitter as a data source for recommendations can have a greater role than that of augmenting recommendations coming from elsewhere. (@TODO More specifically....)

Moving towards a conclusion:

\begin{enumerate}
  \item Twitter's strongest suite lies in its abundance of novel content.
  \item Some of the traditional CF systems' weakest points relate to recommending novel content.
  \item Augmenting new content, with extremely sparse user ratings, might well be a good application.
\end{enumerate}

\section{Suggestions for further work} % (fold)
\label{sec:suggestions_for_further_work}

Suggestions to further work:

\begin{itemize}
  \item Applying NED to Twitter entities.
  \item Further improving sentiment analysis of informal texts.
  \item CRF og sentimentanalyse? -- klassifisere bort tekster? filtrere i et første steg.
  \item Nyere filmer -- tråle IMDB for nye filmer.
  \item
    Gi kontekst til Netflix-ratings:
    (1. legge til adjektiver?, 2. kontrastifisere positive/negative adjektiver. ordsky?)

  \item \emph{Hva vil jeg jobbe med til våren?}
\end{itemize}

% section suggestions_for_further_work (end)
