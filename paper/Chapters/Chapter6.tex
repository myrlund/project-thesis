% Chapter 6

\chapter{Evaluation} % Main chapter title

\label{Chapter6}

\lhead{Chapter 6. \emph{Evaluation}} % This is for the header on each page - perhaps a shortened title

%----------------------------------------------------------------------------------------

% \emph{Beskriv hvordan du vurderer arbeidet ditt. Oppsummer evalueringsresultatene, og bruk dem til å vurdere ditt eget arbeid kritisk. Vær ærlig om eventuelle mangler. Hva betyr resultatene?}

\begin{itemize}
  \item When rating popular and well-known movies\footnote{The sample in question consisted of ``Pulp Fiction'', ``The Shining'', ``Mission: Impossible'', ``The Matrix'', ``The Godfather'', ``Forrest Gump'', and ``A Clockwork Orange''.}
} the predicted ratings achieve a correlational coefficient of 0.75 with regard to average Netflix ratings for the same movie.
  \item B-movies or older less-known movies rarely collect enough Twitter search results to warrant any further analysis.
  \item Movies with titles that are fairly common words or expressions in their own right achieve very low precision, and often return only noise. This is hard to detect without manual interference. Need to perform some sort of Named Entity Disambiguation, maybe something like the techniques outlined in Cucerzan~\cite{NamedEntityDisambiguationWiki} or Sarmento~\cite{NamedEntityDisambiguationWS} (is elaboration needed?).
\end{itemize}

\section{Evaluating against Netflix rating data} % (fold)
\label{sec:evaluating_against_netflix_rating_data}

To find out how the Twitter-based predictions fare, we will use the average of the available Netflix ratings of the same titles as a benchmark.

With predictions $p$, and benchmark ratings $r$, we will compute the MSE (Mean Square Error) of $N$ sample movies in the following way:

\begin{align}
  \text{MSE} &= \frac{1}{N} \sum_{i=1}^N (p_i - r_i)^2
\end{align}

% Good idea?
As a baseline metric, we'll compare against the MSE of the average of all the benchmark ratings:

\begin{align}
  \text{MSE}_\text{baseline} = \frac{1}{N} \sum_{i=1}^N (\bar{r} - r_i)^2
\end{align}

% section evaluating_against_netflix_rating_data (end)
