% Chapter 1

\chapter{Introduction} % Main chapter title

\label{Chapter1} % For referencing the chapter elsewhere, use \ref{Chapter1} 

\lhead{Chapter 1. \emph{Introduction}} % This is for the header on each page - perhaps a shortened title

%----------------------------------------------------------------------------------------

% \emph{Context; motivation for the project; problem statement; outline of dissertation.}

\section{Motivation}

% Movie recommendations have come a long way, but still they seem to be lacking in a few ways.
% 
% Most services supplying automated recommendations
% \begin{enumerate}
%   \item don't provide the context and background for each recommendation
%   \item don't consider public opinion and hype
% \end{enumerate}

Many movie recommendation systems today use collaborative filtering techniques at their core. Although collaborative filtering has many advantages that have let it attain its position as one of the dominant algorithms in the field, there are still quite a few weaknesses left to remedy. Among others, two important challenges for collaborative filtering techniques left today are 1) their ability to handle data sparsity~\cite{Su:2009:SCF:1592474.1722966}, and 2) their ability to explain predictions~\cite{Herlocker:2000:ECF:358916.358995}.

I will take a somewhat untraditional approach in an attempt to mitigate these issues: can data mined from one of today's largest sources of user-generated content, Twitter, contribute enough relevant information to solve both the problem of data sparsity and result explanation?

\subsection{Data sparsity in collaborative filtering}

The data sparsity problem has several sides to it~\cite{Su:2009:SCF:1592474.1722966}.

Here, we will take a closer look at the \emph{cold start} problem -- or more specifically: the \emph{new item problem}. It occurs when a new item enters the system and there is no rating history to base similarity measures on, leaving a barebones collaborative filtering algorithm without anything to base predictions on -- until, of course, some users rate it. If this is the case for a large number of items, we say that we have poor \emph{coverage}.

Second, 

\subsection{Twitter}

Twitter is one of the largest sources of user-generated content available today. It launched in 2006, was incorporated in 2007, and has seen active user growth ever since. At the time of writing, Twitter has more than 230 million registered users sending around 500 million Tweets per day.\footnote{Numbers from \url{https://about.twitter.com/company}.}

One of the most interesting things about Twitter is its simplicity. Each message is limited to 140 characters in length, for no other apparent reason than to force its author to formulate messages very concisely, as well as significantly lower the threshold for publishing content compared to traditional blogging services~\cite{Java:2007:WWT:1348549.1348556}.

Furthermore, 76\% of Twitter's active users are on mobile -- enabling use of the service from anywhere. Combined with Twitter's well-established REST API, this provides us with a robust source of real-time data on almost any subject.

I'll delve further into aspects of using Twitter as a data source in section~\ref{sec:twitter_data}.

% ----------------------------------------------------------------------------------------------------------------------

% Recommendation systems, as seen in commercial products in 2013, seldom provide any context to go with its recommended products other than statements like ``Because you watched X you might like Y.''
% 
% Netflix, the movie subscription service, quite recently began suggesting movies based on what your Facebook friends had watched -- thereby taking a solid step in the direction of social recommendations.
% These suggestions, however, have one underlying assumption that may not always apply:
% what the displayed friend has watched is relevant to your choice of content.
% 
% Furthermore, the social components in most contemporary recommendation systems are \emph{personal}.
% This approach has some clear advantages in that it uses \emph{your} social network as a basis for suggesting content, but there are also some downsides to this:
% 
% \begin{itemize}
%   \item They don't take \emph{novelty} into consideration.
%   \item They don't take \emph{hype} into consideration.
%   \item They leverage only a microscopic portion of the available opinions on content that is available in social media.
% \end{itemize}
% 
% These personal suggestions seems to be a trend in contemporary approaches to social recommendations.
% We'll have a look at something a bit different.

% \subsubsection{A different approach to social media recommendations}

%----------------------------------------------------------------------------------------

\section{Research Questions}

\emph{@REWORK}

In this thesis, I'll mostly examine \emph{improving a set of recommendations} taking social media data into account, and not so much try to generate new recommendations in their own right -- although I might make a go of it if the data should prove agreeable.

I will rather look at ways of using copious\footnote{\emph{Copious} in the sense that the size of the data source is arbitrarily large.}, unpersonal\footnote{\emph{Unpersonal} in the sense that the data is not related to a single hypothetical user of the system.} social media data to \emph{filter} sets of recommendations.

More specifically, the aim is to find answers to the following questions:

\begin{enumerate}
  \item Can sentiment analysis of large quantities of unpersonal social media data be used to effectively filter or provide context to recommendations?
  \item Can unpersonal social media data in any way generate reliable ratings of its own?
  \item How do we evaluate our efforts in order to answer the above questions?
\end{enumerate}

%----------------------------------------------------------------------------------------

\section{Overview and Summary}

This paper is organized as follows.
Chapter~\ref{Chapter2} surveys relevant literature, similar applications, and provides an in-depth analysis of Twitter data and the Twitter API.
Chapter~\ref{Chapter3} 

%----------------------------------------------------------------------------------------
