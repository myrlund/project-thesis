% Chapter 1

\chapter{Introduction} % Main chapter title

\label{Chapter1} % For referencing the chapter elsewhere, use \ref{Chapter1} 

\lhead{Chapter 1. \emph{Introduction}} % This is for the header on each page - perhaps a shortened title

%----------------------------------------------------------------------------------------

% \emph{Context; motivation for the project; problem statement; outline of dissertation.}

\section{Motivation}

% Movie recommendations have come a long way, but still they seem to be lacking in a few ways.
% 
% Most services supplying automated recommendations
% \begin{enumerate}
%   \item don't provide the context and background for each recommendation
%   \item don't consider public opinion and hype
% \end{enumerate}

Recommendation systems, as seen in commercial products in 2013, seldom provide any context to go with its recommended products other than statements like ``Because you watched X you might like Y.''

Netflix, the movie subscription service, quite recently began suggesting movies based on what your Facebook friends had watched -- thereby taking a solid step in the direction of social recommendations.
These suggestions, however, have one underlying assumption that may not always apply:
what the displayed friend has watched is relevant to your choice of content.

Furthermore, the social components in Netflix's recommendations are \emph{personal}.
This approach has some clear advantages in that it uses \emph{your} social network as a basis for suggesting content, but there are also some downsides to this:

\begin{itemize}
  \item They don't take \emph{novelty} into consideration.
  \item They don't take \emph{hype} into consideration.
  \item They leverage only a microscopic portion of the available opinions on content that is available in social media.
\end{itemize}

These personal suggestions seems to be a trend in contemporary approaches to social recommendations.
We'll have a look at something a bit different.

% \subsubsection{A different approach to social media recommendations}

There are other sources of social data that do not fit into the ``personal'' model.
In this thesis we'll have a look at one of the largest sources of unpersonal social media content today: Twitter.

Micro-blogging services such as Twitter have enormous amounts of data on almost every topic imaginable.
Content is limited in length, and users react to each others' content by ``re-tweeting'', ``favoriting'' or ``replying to'' it.
This leaves us with a source of textual data that is:

\begin{description}
  \item[Instant] Users express reactions to events as they experience them.
  \item[Weighted] Users weigh each others' content by interacting with it.
  \item[Concise] Due to limitations on content length, users must express themselves concisely.
\end{description}

Furthermore, the Twitter search API\footnote{\url{https://dev.twitter.com/docs/api/1.1/get/search/tweets}} supports returning both \emph{popular} and \emph{recent} content, or \emph{a mix} of the two.
This enables two interesting approaches to both filtering and annotating the recommended content, in that we can treat popular and recent comments separately.

%----------------------------------------------------------------------------------------

\section{Research Questions}

In this thesis, I'm \emph{not} looking into the task of \emph{generating new recommendations}, but rather trying to \emph{improve a set of recommendations} taking copious social media data into account.

I will rather look at ways of using copious\footnote{\emph{Copious} in the sense that the size of the data source is arbitrarily large.}, unpersonal\footnote{\emph{Unpersonal} in the sense that the data is not related to a single hypothetical user of the system.} social media data to \emph{annotate} and \emph{filter} sets of recommendations.

More specifically, the aim is to find answers to the following questions:

\begin{enumerate}
  \item Can sentiment analysis of large quantities of unpersonal social media data be used to effectively filter recommendations?
  \item Do users want context associated with recommendations? Can filtered unpersonal social media data fill this role?
  \item How do we evaluate our efforts in order to answer the above questions?
\end{enumerate}

%----------------------------------------------------------------------------------------

\section{Overview and Summary}

I will look at improving an intermediate step in a hypothetical recommendation pipeline, filtering and/or annotating recommended content.
For a more detailed look at the system design, and some reasoning around the parts of the system being touched upon, see chapter~\ref{Chapter4}.

\emph{@TODO Add more overview and summary information as it comes into existence.}

%----------------------------------------------------------------------------------------
