% Chapter 1

\chapter{Introduction} % Main chapter title

\label{Chapter1} % For referencing the chapter elsewhere, use \ref{Chapter1} 

\lhead{Chapter 1. \emph{Introduction}} % This is for the header on each page - perhaps a shortened title

%----------------------------------------------------------------------------------------

% \emph{Context; motivation for the project; problem statement; outline of dissertation.}

\section{Motivation}

Many movie recommendation systems today use collaborative filtering techniques at their core. Although collaborative filtering has many advantages that have let it attain its position as one of the dominant algorithms in the field, there are still quite a few weaknesses left to remedy. Among others, two important challenges for collaborative filtering techniques left today are 1) their ability to handle data sparsity~\cite{Su:2009:SCF:1592474.1722966}, and 2) their ability to explain predictions~\cite{Herlocker:2000:ECF:358916.358995}.

I will take a somewhat untraditional approach in an attempt to mitigate these issues: can data mined from one of today's largest sources of user-generated content, Twitter, contribute enough relevant information to solve both the problem of data sparsity and result explanation?

In this project I will attempt to predict movie recommendations based on the perceived sentiment in Twitter search results.

\subsection{Problems in collaborative filtering}

The data sparsity problem has several sides to it~\cite{Su:2009:SCF:1592474.1722966}.

Here, we will take a closer look at the \emph{cold start} problem -- or more specifically: the \emph{new item problem}. It occurs when a new item enters the system and there is no rating history to base similarity measures on, leaving a barebones collaborative filtering algorithm without anything to base predictions on -- until, of course, some users rate it. If this is the case for a large number of items, we say that we have poor \emph{coverage}.

Furthermore, many collaborative filtering algorithms have a problem explaining why they come up with their predictions. If we're able to reliably find Twitter content to back our collaborative filtering predictions, we can use them as a way of providing a context to our recommendations.

\subsection{Twitter}

Twitter is one of the largest sources of user-generated content available today. It launched in 2006, was incorporated in 2007, and has seen active user growth ever since. At the time of writing, Twitter has more than 230 million registered users sending around 500 million Tweets per day.\footnote{Numbers from \url{https://about.twitter.com/company}.}

One of the most interesting things about Twitter is its simplicity. Each message is limited to 140 characters in length, for no other apparent reason than to force its author to formulate messages very concisely, as well as significantly lower the threshold for publishing content compared to traditional blogging services~\cite{Java:2007:WWT:1348549.1348556}.

Furthermore, 76\% of Twitter's active users are on mobile -- enabling use of the service from anywhere. Combined with Twitter's well-established REST API, this provides us with a robust source of real-time data on almost any subject.

I'll delve further into aspects of using Twitter as a data source in section~\ref{sec:twitter_data}.

\section{Research Questions}

In this thesis, I will look for a fit between collaborative filtering's weaknesses and Twitter's strengths. Specifically, I will examine if it is possible to use sentiment extracted from Twitter messages to predict movie ratings.

The main hypothesis is that for each movie, there is a correlation between its user ratings and the sentiment of Twitter messages about it. It is based on a few assumptions:

\begin{enumerate}
  \item Twitter users write positive things about movies they like, and negative things about movies they don't like.
  \item Twitter users express this sentiment within the Twitter messages themselves, not just in linked content.
  \item When a movie is referenced in a Twitter message the title will most likely be mentioned, and spelled correctly.
  \item It is viable to sentimentally classify texts shorter than 140 character, often written in informal language.
  \item It is viable to separate between referenced movies, and other entities or phrases with the same title.
\end{enumerate}

As we shall see, many of these assumptions hold up poorly -- or hardly at all -- rendering the main hypothesis fallacious.

%----------------------------------------------------------------------------------------

\section{Overview}

This paper is organized as follows.

Chapter~\ref{Chapter2} surveys relevant literature, similar applications, provides an in-depth analysis of Twitter data and the Twitter API, and reviews the sentiment classifier chosen for the implementation.
Chapter~\ref{Chapter3} describes the system design, and reasoning behind central design choices.
Chapter~\ref{Chapter4} describes the implementation, specifically the way the APIs are called, as well as central algorithms and how they are employed.
In chapter~\ref{Chapter5} we'll look at the execution results, and see how they evaluate.
Chapter~\ref{Chapter6} summarizes the most important takeaways, and suggests some further work.

%----------------------------------------------------------------------------------------
