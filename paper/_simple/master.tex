\documentclass[11pt,a4paper]{article}

% For å kunne skrive norske tegn.
\usepackage[utf8]{inputenc}

\usepackage{minted}

% For å inkludere figurer.
\usepackage{graphicx}

% Ekstra matematikkfunksjoner.
\usepackage{amsmath,amssymb}

%\usepackage[section]{placeins}

% \usepackage{hyperref}
% \hypersetup{%
%   colorlinks=true, % hyperlinks will be black
%   urlcolor=red,
%   linkcolor=red
% }

% For å få tilgang til finere linjer (til bruk i tabeller og slikt).
%\usepackage{booktabs}

% For justering av figurtekst og tabelltekst.
%\usepackage[font=small,labelfont=bf]{caption}

% Subsections A, B,
%\renewcommand{\thesection}{\Roman{section}}
%\renewcommand{\thesubsection}{\arabic{subsection}}

% Disse kommandoene kan gjøre det enklere for LaTeX å plassere figurer og tabeller der du ønsker.
\setcounter{totalnumber}{5}
\renewcommand{\bottomfraction}{0.95}
\renewcommand{\floatpagefraction}{0.35}

\begin{document}

  % Rapportens tittel:
  \title{Better movie recommendations with copious social media data}
  \author{Jonas Myrlund}

  % Her ber vi LaTeX om å lage tittelen (til nå har vi bare sagt hva den skal inneholde):
  \maketitle
  
  \section{TWERKS} % (fold)
  \label{sec1}
  
  \clearpage
  
  % section sec1 (end)

  \section{TWERKS} % (fold)
  \label{sec2}
  
  % section sec1 (end)

\end{document}















